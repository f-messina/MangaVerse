\chapter{Testing}
Testing is a substantial part of the MangaVerse web application project. Testing helps to ensure application's 
reliability, performance and correctness. To be able to conduct efficient testing process, two kind of tests are preformed.
They are JUnit testing as a structural testing and functional testing.

\section{Structural Testing}
Structural testing also with other name white-box testing is based on testing the internal structure of the working application and it 
guarantees that the methods are working as expected.JUnit testing framework is used to conduct structural testing. JUnit testing is performed by testing 
different modules of the application such as DAOs and services. With that process each methods efficiency and correctness is guaranteed. 
Some examples of JUnit testing are shown below.\\ \\


\textbf{Example code snippet from AnimeDAOMongoImplTest:}
\begin{mdframed}[backgroundcolor=yellow!20, innerleftmargin=10pt, innerrightmargin=10pt]
    \begin{lstlisting}[language=java]
class AnimeDAOMongoImplTest {

@BeforeEach
public void setUp() throws Exception {
    BaseMongoDBDAO.openConnection();
}

@AfterEach
public void tearDown() throws Exception {
    BaseMongoDBDAO.closeConnection();
}
// test 1 : search for an anime by name
// test 2 : search for an anime by filters
@Test
void searchTest() {
    AnimeDAOMongoImpl animeDAO = new AnimeDAOMongoImpl();

    // test 1
    System.out.println("Search by title");
    assertDoesNotThrow(() -> {
        List<MediaContentDTO> animeList = animeDAO.search(List.of(Pair.of("title", "Attack on Titan")), Map.of("title", 1), 1, false).getEntries();
        for (MediaContentDTO anime : animeList) {
            System.out.println("Id: " + anime.getId() + ", Title: " + anime.getTitle());
        }
    });

    // test 2
    System.out.println("Search by filters");
    assertDoesNotThrow(() -> {
        for (int i = 1; i < 5; i++) {
            PageDTO<MediaContentDTO> animePage = animeDAO.search(List.of(Pair.of("$in",Map.of("tags", List.of("school clubs", "manwha")))), Map.of("title", 1), i, false);
            if (!animePage.getEntries().isEmpty()) {
                for (MediaContentDTO anime : animePage.getEntries()) {
                    System.out.println("Id: " + anime.getId() + ", Title: " + anime.getTitle());
                }
            }

        }
    });
}
}
    \end{lstlisting}
\end{mdframed}

\textbf{Example code snippet from Neo4jDAOImplTest:}
\begin{mdframed}[backgroundcolor=yellow!20, innerleftmargin=10pt, innerrightmargin=10pt]
    \begin{lstlisting}[language=java]
public class Neo4JDAOImplTest{

@BeforeEach
public void setUp() throws Exception {
    BaseMongoDBDAO.openConnection();
    BaseNeo4JDAO.openConnection();
}

@AfterEach
public void tearDown() throws DAOException {
    BaseMongoDBDAO.closeConnection();
    BaseNeo4JDAO.closeConnection();
}
@Test
public void testFollowUser() throws DAOException {
    try {
        UserDAONeo4JImpl neo4JDAO = new UserDAONeo4JImpl();
        neo4JDAO.follow("6577877be68376234760585a", "6577877be683762347605859");
    } catch (DAOException e) {
        fail("Exception not expected: " + e.getMessage());
    }

}

@Test
public void testUnlikeAnime() throws DAOException {
    try {
        AnimeDAONeo4JImpl dao = new AnimeDAONeo4JImpl();
        dao.unlike("6577877be68376234760585f","65789bb52f5d29465d0abd09");
    } catch (DAOException e) {
        fail("Exception not expected: " + e.getMessage());
    }

}
}
    \end{lstlisting}
\end{mdframed}


\textbf{Example code snippet from ReviewServiceImpl:}
\begin{mdframed}[backgroundcolor=yellow!20, innerleftmargin=10pt, innerrightmargin=10pt]
    \begin{lstlisting}[language=java]
class ReviewServiceImplTest {
    private static final ExecutorTaskService aperiodicTaskService = ServiceLocator.getExecutorTaskService(ExecutorTaskServiceType.APERIODIC);
    private static final TaskManager errorTaskManager = ServiceLocator.getErrorsTaskManager();
    @BeforeEach
    public void setUp() throws Exception {
        BaseMongoDBDAO.openConnection();
        BaseNeo4JDAO.openConnection();
        aperiodicTaskService.start();
        errorTaskManager.start();
    }

    @AfterEach
    public void tearDown() throws Exception {
        BaseMongoDBDAO.closeConnection();
        BaseNeo4JDAO.closeConnection();
        aperiodicTaskService.stop();
        errorTaskManager.stop();
    }
    @Test
    void updateReview() {
        ReviewServiceImpl reviewService = new ReviewServiceImpl();
        try {
            ReviewDTO reviewAnime = createSampleAnimeReview();
            assertDoesNotThrow(() -> reviewService.addReview(reviewAnime));
            reviewAnime.setComment("This is an updated test review");
            reviewAnime.setRating(4);
            assertDoesNotThrow(() -> reviewService.updateReview(reviewAnime));
            System.out.println("Anime review updated: " + reviewAnime);

            ReviewDTO reviewManga = createSampleMangaReview();
            assertDoesNotThrow(() -> reviewService.addReview(reviewManga));
            reviewManga.setComment("This is an updated test review");
            reviewManga.setRating(4);
            assertDoesNotThrow(() -> reviewService.updateReview(reviewManga));
            System.out.println("Manga review updated: " + reviewManga);
        } catch (BusinessException e) {
            fail(e);
        }
    }
}
    \end{lstlisting}
\end{mdframed}

\newpage

\section{Functional Testing}
Functional testing also with other name black-box testing is based on testing the application's external functionalities. It checks the application from end-user's 
perspective. It ensures that specified requirements are provided efficiently by the web application and expected is outcome is created. 
With the help of the use cases and real world scenarios, functional testing is conducted. Some examples of functional testing are shown below.



\begin{longtable}{|c|>{\RaggedRight}p{2cm}|>{\RaggedRight}p{3.5cm}|>{\RaggedRight}p{3cm}|>{\RaggedRight}p{3cm}|>{\RaggedRight}p{1.5 cm}|}
    \caption{Functional Test Cases} \\
    \hline
    \textbf{Id} & \textbf{Description} & \textbf{Input} & \textbf{Expected Output} & \textbf{Output} & \textbf{Outcome} \\
    \hline
    \endfirsthead

    \hline
    \textbf{Id} & \textbf{Description} & \textbf{Input} & \textbf{Expected Output} & \textbf{Output} & \textbf{Outcome} \\
    \hline
    \endhead

    \hline
    \endfoot

    \hline
    \endlastfoot

    User\_01 & Login with correct information & email: nmiller@example.com, password: f6d6b3ffecb44a... & The user logs in successfully & The user logs in successfully  & PASS \\
    \hline
    User\_02 & Login with wrong information  & email: wrong@example.com, password: wrong  & The user is not able to log in successfully & The user is not able to log in successfully  & PASS  \\
    \hline
    User\_03  & Signup with all the mandatory info filled & username: mikasa, email: mikasa@example.com, password:AttackOn..., gender:female  & The user is registered successfully and logs in  & The user is registered successfully and logs in  & PASS  \\
    \hline
    User\_04  & Signup with missing info & username: eren, password:AttackOn...  & The user is not able to register successfully  & The user is not able to register successfully  & PASS  \\
    \hline
    %signup with already used username or email
    User\_05 & Signup with already used username or email & username:Crystal, email: millerderek@example. com & The user is not able to register successfully & The user is not able to register successfully & PASS \\
    \hline
    User\_06 & Update user information & description: manga lover & User profile is updated with new info.  & User profile is updated with new info.  & PASS  \\
    \hline
    User\_07 & Follow another user & - & User is followed.  & User is followed.  & PASS  \\
    \hline
    %follow a user you already follow
    User\_08 & Follow a user you already follow & - & User is already followed.  & User is already followed.  & PASS  \\
    \hline
    User\_09 & Unfollow another user & - & User is unfollowed. & User is unfollowed.  & PASS  \\
    \hline
    User\_10 & Search manga by title & title: "Slam Dunk"  & The list of manga which includes the words of "Slam Dunk" is shown.  & The list of manga which includes the words of "Slam Dunk" is shown. & PASS  \\
    \hline
    User\_11 & Search manga by detailed filtering & Genres:adventure, comedy, fantasy. Year:2018. Demographics:Josei. Publication Status:finished  & The list of manga including the filtering details is shown  & The list of manga including the filtering details is shown  & PASS  \\
    \hline
    User\_12 & Like anime & - & The anime is liked & The anime is liked  & PASS \\
    %unlike anime
    User\_13 & Unlike anime & - & The like is removed from the anime & The like is removed from the anime  & PASS \\
    \hline
    User\_14 & Add review to anime & review:"I like the anime"  & The review is added to the anime and displayed in the anime page  & The review is added to the anime and displayed in the anime page  & PASS \\
    \hline
    User\_15 & Update review & review: "I dont like this anime anymore" & The review is updated with the new one. & The review is updated with the new one.  & PASS \\
    \hline
    %update review without comment and rating
    User\_16 & Update review without comment and rating & review: "" & The review is not updated. & The review is not updated.  & PASS \\
    \hline
    %delete review
    User\_17 & Delete review & - & The review is deleted. & The review is deleted.  & PASS \\
    \hline
    Admin\_01 & See users distribution analytics   & Distribution type:gender,location, birthday,joined on & The distribution is shown  & The distribution is shown  & PASS  \\
    \hline
    Admin\_02 & See manga analytics to get average rating by month & Year:2020 & Average rating for each month in 2020 is displayed in the page & Average rating for each month in 2020 is displayed in the page  & PASS  \\
    \hline
    Admin\_03 & See anime analytics to get trend media content by year & Year:2023 & List of anime with a larger amount of likes is displayed & List of anime with a larger amount of likes is displayed  & PASS  \\
    \hline
    Admin\_04 & See anime analytics by criteria  & Analytics Type:tags  & The highest average ratings for the tags are shown  & The highest average ratings for the tags are shown & PASS \\
    \hline
\end{longtable}



\section{Performance Testing}

Performance testing is conducted to ensure that MangaVerse web application is able to handle the 
a high volume of operations efficiently and provides a smooth experience for users. 
It is important to test the application's performance to ensure that it 
can handle the expected load. Performance testing is applied on MongoDB and Neo4j databases.
Specifically, the aim of the performance testing is to see the impacts of indexing on CRUD operations
and aggregation operations. 

\subsection*{MongoDB Performance Testing}
Indexes are used in MongoDB to improve the performance of queries. Indexes are used to quickly locate data without having to search every document in a collection.
This limiting the search with indexes, results can get with faster response time.
For the mongoDB performance testing, \textit{username} is used to indexing Users collection and \textit{title} is
used to indexing Anime and Manga collections. The tests are conducted for creating new documents, searching for documents and aggregating documents.
As it can be shown in the table below, there are significant improvement in operation times with indexing, especially for search and aggregation tasks

\begin{longtable}{@{}llcccc@{}}
\caption{MongoDB Performance Test Results} \\
\toprule
\textbf{Collection} & \textbf{Operation} & \textbf{Index} & \textbf{Time (ms)} & \textbf{Total Keys Examined} & \textbf{Total Docs Examined} \\ \midrule
\endfirsthead

\multicolumn{6}{c}%
{{\tablename\ \thetable{} -- continued from previous page}} \\
\toprule
\textbf{Collection} & \textbf{Operation} & \textbf{Index} & \textbf{Time (ms)} & \textbf{Total Keys Examined} & \textbf{Total Docs Examined} \\ \midrule
\endhead

\midrule \multicolumn{6}{r}{{Continued on next page}} \\ \bottomrule
\endfoot

\bottomrule
\endlastfoot

\multirow{6}{*}{Users}
 & Insert & No & 3 & - & - \\
 & Search & No & 17 & 0 & 10007 \\
 & Aggregate & No & 20 & 0 & 10007 \\
 & Insert & Yes & 5 & - & - \\
 & Search & Yes & 5 & 1 & 1 \\
 & Aggregate & Yes & 18 & 1 & 0 \\ \midrule

\multirow{6}{*}{Anime}
 & Insert & No & 3 & - & - \\
 & Search & No & 88 & 0 & 30113 \\
 & Aggregate & No & 32 & 0 & 30113 \\
 & Insert & Yes & 3 & - & - \\
 & Search & Yes & 10 & 0 & 0 \\
 & Aggregate & Yes & 9 & 0 & 0 \\ \midrule

\multirow{6}{*}{Manga}
 & Insert & No & 3 & - & - \\
 & Search & No & 141 & 0 & 41677 \\
 & Aggregate & No & 28 & 0 & 41677 \\
 & Insert & Yes & 3 & - & - \\
 & Search & Yes & 10 & 1 & 1 \\
 & Aggregate & Yes & 12 & 1 & 0 \\

\end{longtable}

\subsection*{Neo4j Performance Testing}

Similarly, also for Neo4j database, performance testing is conducted to observe the impacts of indexing on some CRUD operations.
Indexing with ids is used for both anime, manga and users nodes. The tests are conducted for creating new nodes and searching for nodes.
The test result in the table below shows that indexing has a significant impact on operations in Neo4j database especially for search.
Indexing time decreases with using indexes for search tasks because it prevents to check all the nodes in the database.

\begin{longtable}{@{}llllr@{}}
    \caption{Neo4j Performance Test Results} \\
    \toprule
    \textbf{Collection} & \textbf{Operation} & \textbf{Index} & \textbf{Time (ms)} \\ \midrule
    \endfirsthead
    
    \multicolumn{4}{c}%
    {{\tablename\ \thetable{} -- continued from previous page}} \\
    \toprule
    \textbf{Collection} & \textbf{Operation} & \textbf{Index} & \textbf{Time (ms)} \\ \midrule
    \endhead
    
    \midrule \multicolumn{4}{r}{{Continued on next page}} \\ \bottomrule
    \endfoot
    
    \bottomrule
    \endlastfoot
    
    \multirow{4}{*}{Anime}
     & Insert & No  & 5 \\
     & Insert & Yes & 5\\
     & Search & No  & 45 \\
     & Search & Yes & 2\\ \midrule
    
    \addlinespace
    \multirow{4}{*}{Manga}
     & Insert & No  & 7 \\
     & Insert & Yes & 5 \\
     & Search & No  & 46 \\
    & Search & Yes & 9 \\ \midrule
    
    \addlinespace
    \multirow{4}{*}{Users}
     & Insert & No  & 56 \\
     & Insert & Yes & 9\\
     & Search & No  & 40 \\
     & Search & Yes & 3 \\
    \end{longtable}